\documentclass[platz]{tudphygp}
\usepackage{tudphymd,subfigure}

\versuch{Gekoppelte Schwingungen}{GS}

\begin{document}

\maketitle

\subsection*{Aufgabenstellung}
Das Wissen �ber physikalische Pendel und Schwingungen soll durch diesen Versuch vertieft werden. Messen Sie die Schwebungsfrequenz der gekoppelten Pendel als Funktion der Kopplungsl�nge direkt und indirekt. Die Messergebnisse sind mit den theoretisch berechneten Werten zu vergleichen.

\subsection*{Hinweise zur Versuchsdurchf�hrung}

\begin{enumerate}
 \item Nehmen Sie das \emph{Cassy-Messsystem} in Betrieb. (Nutzer: GS, Passwort: gs)
 \item \emph{ungekoppeltes Pendel:} �berpr�fen Sie die Schwingungsdauern der einzelnen Pendel �ber \num{100} Schwingungen. Entfernen Sie die Feder komplett f�r diese Messung, damit der Einfluss der Masse der Feder entf�llt. Bei Abweichungen die verstellbare Pendelmasse nach R�cksprache mit dem Betreuer verschieben.
 \item \emph{gekoppeltes Pendel:} Messen Sie f�r \emph{drei} verschiedene Koppell�ngen $l$ mit Hilfe des Cassy-Systems:
  \begin{enumerate}
   \item die Schwebungsdauer $\tau$ (Gesamtzeit $t_S$ f�r \num5 bis \num{10} Schwebungen messen)
   \item Schwingungsdauer $T$ der gleichsinnigen Fundamentalschwingung (Gesamtzeit $t_{50}$ f�r \num{50} Schwingungen messen)
   \item Schwingungsdauer $T'$ der gegensinnigen Fundamentalschwingung (Gesamtzeit $t_{50}'$ f�r \num{50} Schwingungen messen)
  \end{enumerate}
 \item[] Hinweis zur Durchf�hrung: Bringen Sie die Koppelfedern sorgf�ltig so an, dass die Pendelspitzenlager nicht in der F�hrung �ber der Hallsonde bewegt werden und immer an den durch Ringmarken gekennzeichneten Positionen verbleiben.
 \item Messen Sie die Federkonstante $k$ der Kopplungsfeder mittels dynamischer Methode (wie?).
 \item[] Hinweis zur Durchf�hrung: Verwenden Sie eine hinreichend gro�e Masse. Dabei darf die Feder allerdings nicht zu weit gedehnt werden, da sich die Federkonstante bei gro�en Auslenkungen (u.~U. bleibend) �ndert.
\end{enumerate}

\subsection*{Zur Benutzung des Cassy-Messsystems}

\begin{itemize}
 \item �berpr�fung der Schwingungsdauern: W�hlen Sie im Cassy-Datenerfassungsprogramm eine geeignete Achseneinteilung f�r die Hallspannung (z.~B. zwischen \num{-0,5} und \SI{0.5}V). Stellen Sie mit Hilfe des Offsetreglers an den Hallsonden f�r die Ruhelage der Pendel etwa \SI0V ein. Die Datenaufnahme sollte zun�chst bei \SI{0,1}s liegen. Arbeiten Sie immer im Modus "`Automatische Aufnahme"'.
 \item Bestimmung der Schwebungsdauer: Bei sehr kleinen Schwebungsdauern ist das Zeitintervall der Datenaufnahmen auf \SI{0.05}s zu reduzieren.
 \item Zur Ermittlung der Zeiten $t_S$, $t_{50}$ und $t_{50}'$: Entweder lesen Sie die Zeiten aus der tabellarischen Darstellung der Hallspannungen ab, oder Sie nutzen das Messtool "`Markierung setzen -- Differenz Messen"'.
\end{itemize}

\subsection*{Auswertung}

\begin{enumerate}
 \item Drucken Sie die erhaltenen Messdiagramme f�r die gleichsinnige bzw. die gegensinnige Fundamentalschwingung und f�r die Schwebung jeweils zur \emph{gr��ten} Koppell�nge $l$ mit der entsprechenden Beschriftung (\texttt{Alt+T}) aus.
 \item Berechnen Sie die Schwebungsfrequenz:
  \begin{enumerate}
   \item $f_{S,1}$: \emph{direkt} aus der der Schwebungsdauer $\tau$
   \item $f_{S,2}$: \emph{indirekt} aus den Schwingungsdauern der beiden Fundamentalschwingungen $T$ und $T'$
   \item $f_{S,3}$: \emph{theoretisch} aus der Federkonstante $k$ und den Parametern des Versuchplatzes (s.~Tab.~\ref{tab::platz})
  \end{enumerate}
  \[
   f_{S,1/2/3} = \frac1\tau = \frac1{T'}  - \frac1T = \frac1T \cdot \kla{\sqrt{1 + \frac{2 \cdot k \cdot l^2}{m \cdot g \cdot s_A}} - 1}
  \]
 \item Stellen Sie die drei Schwebungsfrequenzen in einem \emph{gemeinsamen} Diagramm dar.
 \item Berechnen Sie f�r alle drei Methoden die \emph{relativen} Maximalfehler f�r die kleinste und die gr��te Koppell�nge.
 \item Tragen Sie die Fehlerbalken der jeweiligen \emph{absoluten} Maximalfehler in das Diagramm ein.
\end{enumerate}

\begin{table}\centering
 \begin{tabular}{cccc}
  Messplatz & Hallsondennummer & $m / \SI{}{kg}$       & $s_A / \SI{}{cm}$  \\\hline
  a         & 026 \& 030       & \errnum{1,322}{0,005} & \errnum{87,0}{0,5} \\
  b         & 023 \& 025       & \errnum{1,316}{0,005} & \errnum{87,2}{0,5} \\
  c         & 027 \& 022       & \errnum{1,310}{0,005} & \errnum{87,2}{0,5} \\
  d         & 028 \& 024       & \errnum{1,318}{0,005} & \errnum{87,1}{0,5}
 \end{tabular} \hspace{4em}
 \begin{tabular}{cc}
  $l / \SI{}{cm}$ \\\hline
  \errnum{28,0}{0,2} \\
  \errnum{52,8}{0,2} \\
  \errnum{77,8}{0,2}
 \end{tabular}
 \caption{Parameter f�r die einzelnen Messpl�tze und Koppell�ngen an den drei Ringmarken}\label{tab::platz}
\end{table}

\subsection*{Hinweise f�r das Protokoll}

\begin{table}[hb]\centering
 \subtable[Bestimmung der Federkonstante]{
  \begin{tabular}{c|c|c|c}
   $m / \SI{}g$ & $T_{100,\mathrm{Feder}} / \SI{}s$ & $T_\mathrm{Feder} / \SI{}s$ & $k / \SI{}{N.m^{-1}}$ \\\hline
   \vdots&\vdots&\vdots&\vdots
  \end{tabular}
 }\vspace{2em}
 \subtable[Bestimmung der Schwebungfrequenz]{
  \begin{tabular}{c|cc|cc|cc|ccc}
   $l / \SI{}{cm}$ & $t_{50} / \SI{}s$ & $T / \SI{}s$ & $t_{50}' / \SI{}s$ & $T' / \SI{}s$ & $t_S / \SI{}s$ & $\tau / \SI{}s$ & $f_{S,1} / \SI{}{Hz}$ & $f_{S,1} / \SI{}{Hz}$ & $f_{S,1} / \SI{}{Hz}$ \\\hline
   \vdots&\vdots&\vdots&\vdots&\vdots&\vdots&\vdots&\vdots&\vdots&\vdots
  \end{tabular}
 }
 \caption{Empfohlene Tabellen f�r das Protokoll}
\end{table}

\end{document}