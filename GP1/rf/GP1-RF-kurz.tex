\documentclass[platz]{tudphygp}
\usepackage{tudphymd,booktabs,boxedminipage}

\versuch{Innere Reibung von Fl�ssigkeiten}{RF}
\author{Dr. Schwab, Dr. Escher}

%\forceitem soll sich wie \item verhalten, nur ohne itemize drum herum
\newlength\itempos
\newcommand\forceitem[1][\labelitemi]{
 \setlength\itempos{\labelsep} %eigentlich �ber \itemindent-\labelsep zu berechen, aber \itemindent = 0pt
 \hspace*{\itempos}#1\hspace*{\labelsep}
}

\begin{document}

\maketitle

\subsection*{Aufgabenstellung}

\begin{enumerate}
 \item Bestimmen Sie die Abh�ngigkeit der Z�higkeit $\eta$ des Ethanols von der Temperatur $T$. Dazu sind f�r f�nf verschiedene Temperaturen zwischen Raumtemperatur und \SI{50}{\degC} je zwei Messungen durchzuf�hren.
 \item Geben Sie die Koeffizienten $A$ und $b$ der \emph{Andradeschen Gleichung}
 \item \emph{(F�r Physikstudenten)} Geben Sie die Standardabweichungen der Andradeschen Koeffizienten an. Berechnen Sie die \emph{Reynoldssche Zahl} $Re$ f�r die h�chste Temperatur.
 \item Bestimmen Sie aus der grafischen Darstellung des nat�rlichen Logarithmus der Viskosit�t ($\ln \eta$) �ber dem Reziproken der Temperatur ($1/T$) den Maximalfehler des Anstiegs ($\Delta b$). Tragen Sie dazu die Fehlerfl�chen der Messwerte sowie die Geraden minimalen und maximalen Anstiegs in die grafische Darstellung ein.
 \item \emph{(F�r Physikstudenten)} �berpr�fen Sie anhand der berechneten \emph{Reynoldsschen Zahl}\footnote {O.~Reynolds 1842-1912}, ob die Str�mung laminar war.
 \item Diskutieren Sie den Einfluss der fehlerhaften Gr��en auf den Fehler von $b$.
\end{enumerate}


\subsection*{Hinweise zur Versuchsdurchf�hrung}

\begin{enumerate}
 \item Die Viskosit�t bestimmen Sie aus den Messdaten des \emph{H�ppler-Viskosimeters} nach:
 \begin{refequation}{8}
  \eta = K \cdot (\rho_k - \rho_f)\cdot g\,t
  \label{rf07}
 \end{refequation}%
 %TODO: Gibt es �berhaupt einen Versuchspaltz mit diesem Ger�t?
 \item Beim Viskosimeter nach \emph{Ubbelohde} gilt:
 \begin{refequation}{10}
  \eta = C\,\rho_f\,g\cdot (t - t_k) \label{rf09}
 \end{refequation}%
 \item Die \emph{Andradesche Gleichung} f�r die Temperaturabh�ngigkeit der Viskosit�t lautet:
 \begin{refequation}{2}
  \eta(T)=A\cdot \mathrm{e}^{\frac{b}{T}}
 \label{rf02}
 \end{refequation}%
 \item Die \emph{Reynoldssche Zahl} bestimmen Sie mit:
 \begin{refequation}{11}
  Re = \frac{\rho\cdot v\cdot l^*}{\eta}
  \label{rf10}
 \end{refequation}%
 Dabei gilt f�r die Kugel beim Viskosimeter nach \emph{H�ppler} $l^* = 2R$, der kritische (nicht zu �berschreitende) Wert ist $ Re_\mathrm{krit} \approx 1000$. F�r die Anordnung beim Viskosimeter nach \emph{Ubbelohde}  gilt $l^* = d$, der kritische Werte betr�gt $Re_\mathrm{krit}\approx 2200$.
\end{enumerate}

\newpage
\subsection*{Ger�te- und Umgebungsparameter}

Der analytische Ausdruck f�r die Dichte des Ethanols in Abh�ngigkeit von der Temperatur lautet:
\[
 \rho(\vartheta) = \kla{\num{-8e-4}\cdot \frac\vartheta{\SI{}{\degC}} + \num{,8063} }\cdot \SI{}{g.cm^{-3}}
\]

\begin{table}[h!t]
 \begin{boxedminipage}{\linewidth} \centering
  \begin{tabular}{ccccc}
   Messplatz & $V$ / [\SI{}{cm^3}] & $r$ / [\SI{}{mm}] & $K$ / [\SI{}{mm^2.s^{-2}}] & $t_k$ \\ \toprule
   a         & \num{5,73}        & \num{0,7}       & \num{0,1005}                         & $\num{227,4} \; \SI{}{s^3} / t^2$ \\
   b         & \num{5,73}        & \num{0,7}       & \num{0,09982}                        & $\num{227,4} \; \SI{}{s^3} / t^2$
  \end{tabular}
  \caption{Konstanten und Ger�teparameter f�r die Viskosimeter nach \emph{Ubbelohde}\newline
          Korrekturzeiten in Abh�ngigkeit von der Messzeit \newline
          $K = C \cdot g$ \newline
          $r$ = Radius des Rohres
         }
 \end{boxedminipage}
\end{table}

\begin{table}[h!t]
 \begin{boxedminipage}{\linewidth} \centering
  \begin{tabular}{c|cc|cc|cc}
   Messplatz & $K_\text{vorw�rts}$ / & $K_\text{r�ckw�rts}$ / & $\rho_k$ /         & $\Delta \rho_k$ /  & $d$ /       & $\Delta d$ / \\
             & [\SI{}m]              & [\SI{}m]               & [\SI{}{g.cm^{-3}}] & [\SI{}{g.cm^{-3}}] & [\SI{}{mm}] & [\SI{}{mm}]  \\\toprule
   c         & \num{0,00937e-7}      & \num{0,00923e-7}       & \num{2,4}          & \num{0,01}         & \num{15,81} & \num{0,01}   \\
   d         & \num{0,01123e-7}      & \num{0,01113e-7}       & \num{2,41}         & \num{0,01}         & \num{15,80} & \num{0,01}
  \end{tabular}
  \caption{Konstanten und Ger�teparameter f�r die Viskosimeter nach \emph{H�ppler}\newline
          Stellung des Viskosimeters: \newline
          \forceitem $K_\text{vorw�rts}$ = Zuleitung zum Thermostat ist \emph{unten}; \newline
          \forceitem $K_\text{r�ckw�rts}$ = Zuleitung zum Thermostat ist \emph{oben}; \newline
          $d$ = Kugeldurchmesser (Typ 1G-1)
         }
 \end{boxedminipage}
\end{table}

\end{document}
