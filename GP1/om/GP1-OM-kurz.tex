\documentclass[platz]{tudphygp}
\usepackage{tudphymd}

\versuch{Oszilloskop - Messtechnik}{OM}

\begin{document}
\maketitle

\section*{Hinweise zum Protokoll}

Ein Protokoll muss \emph{alle} Informationen enthalten, die zur Wiederholung des Versuchs unter gleichen
Bedingungen notwendig sind. Tragen Sie \textbf{alle Einstellungen} und \textbf{alle abgelesenen Werte} sofort in
Ihr Protokollheft ein. Geben Sie dazu immer die \textbf{Ma�einheit} und den \textbf{Fehler} (Messgenauigkeit)
an. Auch Hilfsrechnungen und Zwischenergebnisse sind zum Nachvollziehen Ihrer Analyse wichtig.
Notieren Sie also z.B. auch Kennbuchstaben des Versuchsplatzes, Typ der Messger�te, usw. .

\section{Inbetriebnahme der Ger�te}

Schalten Sie die Ger�te ein und verbinden Sie den Funktionsgenerator mittels Koaxkabel
mit dem Kanal 1 (\texttt{CH 1}) des DSO.

\section{Grundmessungen mit dem Funktionsgenerator}

\begin{enumerate}
 \item Stellen Sie den Funktionsgenerator auf \emph{Sinusfunktion ohne Offset} mit einer Frequenz
 $f = \SI{1,2}{kHz}$, dr�cken Sie \texttt{AUTOSET} am DSO und notieren Sie die auf dem Display
 angezeigten Parameter der Sinusspannung.
 \emph{Dokumentieren} Sie die Einstellungen des Vertikalmen�s (Men� \texttt{CH 1}) und des Triggermen�s sowie den Erfassungsmodus.
 \item Stellen Sie nun mit Hilfe des Men�s Messung die Amplitude ein: $U_m = \SI{200}{mV}$.
 Machen Sie sich mit der Vertikal- und Horizontalskalierung vertraut und dokumentieren
 Sie den Einfluss der Triggerschwelle!
 \item Stellen Sie am Funktionsgenerator die \emph{Offsetspannung} auf Maximum (rechter Anschlag)
 und schalten Sie den Offset ein, was beobachten Sie?
 
 Stellen Sie den Trigger so ein, dass auf dem Display eine auswertbare Darstellung entsteht.
 Welche M�glichkeiten der Messung der Offsetspannung bietet das DSO?
 
 W�hlen Sie die Methode, die Ihnen den kleinsten Fehler liefert und stellen Sie das
 Ergebnis mit Fehlergrenze dem einer direkten Messung mit dem Multimeter \texttt{FLUKE 175}
 [Messunsicherheit: $\Delta U = \pm (\num{0,0015}\cdot U + 2 \mathrm{Digit})$] gegen�ber.
 \item Stellen Sie die vom Funktionsgenerator gelieferten \emph{Dreieck-} und \emph{Rechteckspannungen}
 dar, messen und dokumentieren Sie die Parameter dieser Signale mit Hilfe der Funktionen
 im Men� \texttt{MESSUNG}.
 \item Aktivieren Sie den \texttt{CH 2} und verbinden Sie diesen Eingang mit dem \emph{Triggerausgang} des
 Funktionsgenerators, vergleichen Sie dieses Triggersignal mit den Signalen des Generators.
 Was zeichnet dieses Triggersignal aus?
 
 W�hlen Sie als Triggerquelle \texttt{Ext.} und nutzen Sie jetzt die externe Triggerquelle.
\end{enumerate}


\section{Identifizierung von Signalen des Signalgenerators}

\begin{description}
 \item [Achtung:] Messen und protokollieren Sie die wichtigsten Parameter der Signale sowie die Erfassungsart und dem Triggermodus.
\end{description}

\begin{enumerate}
 \item Identifizieren Sie die Signale am BNC-Ausgang bei den\emph{Wahlschalter-Positionen} (WP) 1 bis 3
 und messen Sie alle relevanten Parameter.
 \item Bestimmen Sie die Frequenz der Sinusspannung am separaten Ausgang (BNC, links) und
 �berpr�fen Sie das Ergebnis mit dem durch Frequenzvergleich mit dem Sinussignal des
 \texttt{HM8030 - 5} im \emph{XY-Betrieb} des DSO und \emph{Lissajous-Figuren} erhaltenen Wert.
 \item Messen Sie die Phasendifferenz zwischen dieser Sinusspannung und der bei WP 4 �ber eine
 Zeitdifferenzmessung.
 \item \textbf{Anwendung des passiven Tastkopfes}\\
 Bringen Sie die Rechteckimpulsfolge WP 7 zur Darstellung, skizzieren Sie die Impulsform,
 schlie�en Sie nun den DSO �ber den Tastkopf, Tastteilung $1:1$, an, skizzieren Sie die
 Impulsform erneut und bestimmen Sie alle wichtigen Parameter.
 Wiederholen Sie diese Darstellung und Messung mit dem Teilerverh�ltnis $10:1$.
 
 \textbf{Wichtig:} Achten Sie auch auf die Einstellung des Teilerverh�ltnisses im Vertikalmen�!
 
 Worauf sind die Unterschiede in den Impulsabbildungen zur�ckzuf�hren?
 \item F�hren Sie alle weiteren Messungen mit dem Tastkopf durch!\\
 Welches Teilerverh�ltnis ist dabei zweckm��ig?
 \item Identifizieren Sie die Signale bei den Wahlschalter-Positionen 5 und 8 sowie das Singleshot-Signal
 WP 6. Skizzieren Sie diese Signale und geben Sie jeweils alle aussagekr�ftigen Parameter an.

\end{enumerate}

\end{document}
