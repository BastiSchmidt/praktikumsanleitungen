\documentclass[platz]{tudphygp}
\usepackage{multirow}

\versuch{Dichte von Fl�ssigkeiten}{DF}

\begin{document}
\maketitle

\subsection*{Aufgabenstellung}

\begin{enumerate}
 \item Ermittlen Sie die Genauigkeit der Massenbestimmung mit einer elektronischen Waage.
 \item Bestimmen Sie die Dichte eines Festk�rpers nach dem \emph{Absolutverfahren}.
 \item Bestimmen Sie die Dichte eines Festk�rpers nach dem \emph{Relativverfahren}.
 \item Bestimmen Sie die Dichte einer Fl�ssigkeit mit Hilfe eines \emph{Pyknometers}.
 \item Vergleichen Sie die Genauigkeit der verwendeten Verfahren.
\end{enumerate}

\subsection*{Genauigkeit der Massenbestimmung}

\begin{enumerate}%TODO: "zehn" nach eigenem Ermessen hingeschrieben (original: "wiederholt")
 \item Wiegen Sie eine Materialprobe \emph{zehn} mal und geben Sie $m \pm \Delta m$ an. (Ber�cksichtigen Sie systematische und zuf�llige Fehler.)
 \item F�r alle weiteren Messungen ermitteln Sie den Kallibrierfehler mittels einer bekannten Masse.
\end{enumerate}

\subsection*{Festk�rperdichte mittels Absolutverfahren}

\begin{enumerate}
 \item Wiegen Sie die Masse eines zylindrischen Probek�rpers mittles elektronischer Waage.
 \item Messen Sie den Durchmesser und die H�he des Zylinders und gewinnnen Sie daraus das Volumen.
 \item Berechnen Sie \emph{direkt} aus dem Volumen und der Masse die Dichte und deren Fehler: $\varrho = \frac mV$
\end{enumerate}

\subsection*{Festk�rperdichte mittels Relativverfahren}

\begin{enumerate}
 \item Nutzen f�r die Messung den \emph{Jolly-Aufsatz}.
 \item Messen Sie die Masse des Festk�rpers einmal auf der Schale, die in das Wasser taucht ($m_\mathrm{Fl}$) und einmal auf der Schale, die nicht in das Wasser taucht ($m_\mathrm{Luft}$).
 \item Berechnen Sie die Dichte und deren Fehler:
  \[
   \varrho_x = \frac{m_\mathrm{Luft}}{m_\mathrm{Luft} - m_\mathrm{Fl}} \cdot \varrho_0
  \]
  N�herung der Luftdichte im Bereich zwischen \num{10} und \SI{30}{\degC} (Temperaturen $\vartheta$ in \degC{} einsetzen):
  \[
   \varrho_0 = \frac{\num{999,8395} + \num{16,95258} \cdot \vartheta - \num{7,9905e-3}\cdot \vartheta^2}{1 + \num{1,6887e-2} \cdot \vartheta} \cdot \num{e-3} \cdot \SI{}{g.cm^{-3}}
  \]
\end{enumerate}

\subsection*{Fl�ssigkeitsdichte mittels Pyknometer}

\begin{enumerate}
 \item Messen Sie (Reihenfolge �berlegen!): Leermasse $m_P$ des Pyknometers, Masse $m_1$ des Pyknometers gef�llt mit Wasser, Masse $m_2$ des Pyknometers mit zu untersuchender Fl�ssigkeit gef�llt.
 \item Berechnen Sie die Dichte und deren Fehler nach:
  \[
   \varrho_x = \frac{m_x}{m_0} \cdot \varrho_0 = \frac{m_2 - m_P}{m_1 - m_P} \cdot \varrho_0
  \]
\end{enumerate}

\subsection*{Hinweise zum Versuch}

\begin{itemize}
 \item W�gest�cke sind mit einer Pinzette mit Plastspitzen anzufassen.
 \item Legen Sie den zu w�genden K�rper nach Nullabgleich der Waage vorsichtig m�glichst in die Mitte der Schale.
 \item Bei der Ausmessung des Testk�rpers mit dem Messschieber sind die Messungen an verschiedenen Stellen vorzunehmen, um einen reellen Mittelwert zu erhalten.
 \item Vermeiden Sie bei der W�gungen in oder von Fl�ssigkeiten die Bildung von Luftblasen.
 \item Bei Messungen mit dem Pyknometer ist die exakte F�llh�he zu gew�hrleisten und Verdunstungsverluste m�ssen durch Abdeckung der Kapillare mit einem Deckgl�schen reduziert werden.
 \item Das Pyknometer ist vor der Verwendung zu trocknen, damit exaktes Leergewicht erreicht bzw. eine Vermischung von Fl�ssigkeiten vermieden wird.
 \item Achten Sie auf Konstanz der Temperatur w�hrend der Messung mit einem Pyknometer.
\end{itemize}

\subsection*{Zusammenstellung der systematischen Fehler der verwendeten Messmittel}
\begin{itemize}
 \item \emph{Elektronische Waage}: siehe Datenblatt
 \item \emph{W�gest�cke}: sie gen�gen der OIML-Norm\footnote{OIML: Organisation Internationale de M\'etrologie L\'egale} und sind \emph{Feingewichtsst�cke} der Klasse F1.
  \begin{table}[h!t]\centering
   \begin{tabular}{l|ccccccccccc}
    Nennmasse  & \SI1{kg} & \SI{500}g & \SI{200}g & \SI{100}g & \SI{50}g & \SI{20}g & \SI{10}g & \SI5g & \SI2g & \SI1g & \SI{500}{mg}\\\hline
    $\Delta m / \SI{}{mg}$ & \num{5,0} & \num{2,5} & \num{1,0} & \num{0,5} & \num{0,3} & \num{0,25} & \num{0,2} & \num{0,15} & \num{0,12} & \num{0,1} & \num{0,05}
   \end{tabular}
   \caption{Zur verwendende Fehler der W�gest�cke}\label{tab::errMass}
  \end{table}
 \item \emph{Pyknometer}: hat die Genauigkeitsklasse B
  \begin{table}[h!t]\centering
   \begin{tabular}{l|ccccc}
    $V_n / \SI{}{ml}$ & \num1 & \num5 & \num{10} & \num{25} & \num{50} \\\hline
	$\Delta V / \SI{}{ml}$ & \num{0,003} & \num{0,003} & \num{0,005} & \num{0,010} & \num{0,020}
   \end{tabular}
   \caption{Volumenfehler f�r Pyknometer mit dem Nenninhalt $V_n$}
  \end{table}
 \item \emph{Laborthermometer}:
  \begin{table}[h!t]\centering\def\cc#1{\multicolumn1c{#1}}
   \begin{tabular}{lc|rrrr}
    \multicolumn{2}{l|}{Skaleneinteilung in \SI{}K}                         & \cc{\num1}&\cc{\num{0,5}}&\cc{\num{0,2}}&\cc{\num{0,1}}\\\hline
	\multirow{5}{*}{Messbereich in \SI{}\degC} & $\num{-5}\ldots \num{60}$  & \SI{0,7}K & \SI{0,5}K & \SI{0,2}K & \SI{0,15}K \\
	                                           & $\num{60}\ldots\num{110}$  & \SI1K     & \SI{0,5}K & \SI{0,3}K & \SI{0,25}K \\
	                                           & $\num{110}\ldots\num{210}$ & \SI{1,5}K & \SI1K     & \SI{0,5}K &            \\
	                                           & $\num{210}\ldots\num{310}$ & \SI{2}K   & \SI{1,5}K &           &            \\
	                                           & $\num{310}\ldots\num{400}$ & \SI{2,5}K &           &           &            \\
   \end{tabular}
  \caption{Messfehler der Thermometer abh�ngig vom Temperaturbereich und von der Skaleneinteilung}
  \end{table}
 \item \emph{Messschieber}: Messfehler $\Delta l$ f�r die gemessenen L�nge $l$: $\Delta l = \SI{50}{�m} + \num{e-4} \cdot l[\SI{}{�m}]$
\end{itemize}

\end{document}