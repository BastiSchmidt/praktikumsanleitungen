\documentclass[platz]{tudphygp}
\usepackage{tudphymd,mhchem}

\versuch{Adiabatische Zustands�nderungen}{AZ}

\begin{document}
\maketitle

\subsection*{Aufgabenstellung} 

Der Adiabatenexponent $\kappa = \frac{c_p}{c_v}$ von Luft ist auf folgende Arten zu bestimmen
\begin{enumerate}
 \item aus den Eigenfrequenzen einer mit offenen bzw. geschlossenen H�hnen schwingenden \ce{Hg}-S�ule
 \item nach der Methode von \textsc{Cl\`ement-Desormes}
\end{enumerate}

\section*{Hinweise zur Versuchsdurchf�hrung}

\subsection*{Schwingende Quecksilbers�ule}

\begin{enumerate}
  \item Zur Bestimmung der Resonanz-Schwingungsdauern $T_\mathrm{o}^*$ (offene) und $T_\mathrm{zu}^*$ (geschlossene H�hne)
 werden die U-Rohre in beiden F�llen mit Hilfe eines Motor-Antriebes zu erzwungenen
 Schwingungen angeregt und zwei Resonanzkurven aufgenommen. Hierzu misst man die Amplitude der
 erzwungenen Schwingung als Funktion der Anregungsfrequenz jeweils an sechs Messpunkten um die
 Resonanzpunkte $T_\mathrm o^*$ und $T_\mathrm{zu}^+$ herum. Dabei wird f�r die jeweilige vom Motor vorgegebene
 Frequenz die Schwingungsdauer �ber \num{50} Perioden $T$ gemittelt. Die den beiden
 Resonanzpunkten zugeordneten Schwingungsdauern $T_\mathrm o^*$ und $T_\mathrm{zu}^*$ werden grafisch bestimmt und
 unter Vernachl�ssigung der D�mpfung zur Auswertung benutzt.
  \begin{figure}[ht]
  \centering
  \includegraphics[]{GP1-AZ-URohrn}
  \caption{Kompletter Versuchsaufbau der Schwingenden Quecksilbers�ule \label{fig::Aufbau}}
 \end{figure}
\item Die Frequenz wird mit der Drehzahlfeinregelung (Helipot mit 10 Umdrehungen) des Getriebemotors sehr genau eingestellt. Siehe Abbildung \ref{fig::Aufbau}
 \end{enumerate}
 \begin{enumerate}\setcounter{enumi}{2} %wenn das enumerate hier weitergeht, sind die items wegen dem parpic negativ einger�ckt
 \item Der Hahn 3 dient dem Luftdruckausgleich mit der Umgebung vor der Messung mit geschlossenen H�hnen 1 und 2 und wird vor dieser Messung nur kurzzeitig ge�ffnet und bleibt sonst geschlossen.
 \item Die dem �u�eren Luftdruck entsprechende H�he $h_L$ der Quecksilbers�ule wird mit dem
 Quecksilberbarometer im Raum D113 gemessen.
  \item Tragen Sie die Amplitude $X$ �ber der Schwingungsdauer $T_1$ auf, verwenden Sie dazu die veinfachte Form:
 \begin{refequation}{10}
  X \sim \frac{T_1}{T_2}
 \end{refequation}
 \item Der Adiabatenexponent folgt aus der Gleichung:
 \begin{refequation}{9}
  \kappa = \frac{V}{A h_L} \brk{\frac{T_o^{*2}} {T_{zu}^{*2}} - 1}
 \end{refequation}
\end{enumerate}

\subsection*{Cl\`ement-Desormes}

F�hren Sie zehn Messungen nach folgendem Schema durch:
\begin{enumerate}
 \piccaption{Aufbau f�r den Versuch nach Cl\`ement-Desormes}
 \parpic[r]{
  \includegraphics[width=\linewidth/3]{GP1-AZ-Clement}
 }
 \item Nachdem der Federbolzen gespannt wurde, wird mit Hilfe der Handpumpe der Luftdruck in dem verschlossenen Gef�� um $\Delta p = \rho_w \cdot g h_1$
 (ca. \SI{15}{cm} Wassers�ule) \emph{erh�ht} und dann das Ventil geschlossen. Dabei erh�ht
 sich die Temperatur der Luft in dem Gef�� leicht.
 \item Danach wird ca. \emph{zwei Minuten} gewartet, bis die Temperatur im Gef�� wieder auf \emph{Raumtemperatur}
 abgesunken ist und der Druck konstant bleibt.
 \item Nun wird $h_1$ abgelesen und notiert.
 \item Jetzt wird der Ausl�ser kurz gedr�ckt, so dass �ber den zur�ckschnellenden Federbolzen der Beh�lter f�r eine konstante kurze Zeit ge�ffnet und \emph{schnell} wieder
 verschlossen wird. Bei der adiabatischen Expansion hat sich das Gasvolumen abgek�hlt und es
 erw�rmt sich nun wieder bis zur Raumtemperatur, wonach der Druck wiederum konstant bleibt.
 \item Zum Schluss wird $h_2$ abgelesen.
 \item Der Adiabatenexponent $\kappa$ folgt aus der Beziehung:
 \begin{refequation}{11}
  \kappa = \frac{h_1}{h_1-h_2}
 \end{refequation}
\end{enumerate}

\subsection*{Ger�teparameter}

\begin{table}[h!b]
 \centering
 \begin{tabular}{|ccc|}\hline
      & Platz 1 & Platz 2 \\\hline
  $V$ & \errSI{106,7}{,5}{cm^3} & \errSI{118,2}{,5}{cm^3} \\
  $A$ & \errSI{62,1}{,6}{mm^2} & \errSI{54,7}{,1}{mm^2}\\\hline
 \end{tabular}
 \caption{Relevante Abmessungen des U-Rohrs am jeweiligen Versuchsplatz. Die Volumina in den Schenkeln eines U-Rohres sind jeweils gleich.}
\end{table}

\end{document}
