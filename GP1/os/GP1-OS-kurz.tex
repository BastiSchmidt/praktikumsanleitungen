\documentclass[platz]{tudphygp}
\usepackage{tudphymd}

\versuch{Oberfl�chenspannung}{OS}

\begin{document}
\maketitle

\subsection*{Aufgabenstellung}
Kalibrieren Sie eine Torsionswaage und bestimmen Sie mit dem Abrei�verfahren die Oberfl�chenspannung einer Fl�ssigkeit.

\subsection*{Hinweise zur Versuchsdurchf�hrung}
\begin{itemize}
 \item \textbf{Kalibrierung:} Stellen Sie die Nulllage der Torsionswaage ein.
 \item Eichen Sie die Waage mit bekannten Massest�cken. Stellen Sie dazu bei jeder Messung die Gleichgewichtslage her. Die Gewichte h�ngen Sie immer in einer Kerbe ein (warum?). Vor jeder Messung sollten Sie die Nulllage kontrollieren und ggf. neu einstellen.
 \item Stellen Sie die Kalibrierungskurve $F(\alpha)$ mit Fehlerkreuzen grafisch dar.
 \item \textbf{Bestimmung der Oberfl�chenspannung:} Messen Sie die Abrei�kraft mit sechs verschiedenen B�geln. (Die Drahtl�nge $l$ bestimmen Sie mit einem Lineal.) Tauchen Sie dazu den B�gel ca.~\SI1{mm} unter die Wasseroberfl�che ein. Justieren Sie die Fu�schrauben des Statives nach, damit der B�gel immer parallel zur Oberfl�che ist (zur Kontrolle das Spiegelbild nutzen). Bestimmen Sie $\alpha_1$ und daraus $F_1$, die B�gelgewichtskraft abz�glich der Auftriebskraft der eingetauchten B�gelteile.
 \item Ziehen Sie den B�gel bis zum Abriss heraus. Bestimmen Sie $\alpha_2$ und daraus $F_2$. F�r diese Kraft gibt es eine obere Grenze (Abriss erfolgt sofort) und eine untere Grenze (Abriss erfolgt nach einer gewissen Wartezeit). Dieses Verhalten soll in der Fehlerdiskussion mit besprochen werden.
 \item Die Oberfl�chenspannung bestimmen Sie grafisch aus dem Anstieg der $F(l)$-Kurve:
  \begin{refequation}{6}
   F_\sigma = \sigma \cdot 2l + \mathcal O
  \end{refequation}
\end{itemize}
Beachten Sie f�r die Auswertung die Hinweise in der ausf�hrlichen Anleitung!

\begin{table}[b]\centering
 \begin{tabular}{c|c}
  \textbf{Fl�ssigkeit} & $\sigma / \SI{}{Nm^{-1}}$ \\\hline
  Aceton        & \num{0,0233} \\
  Wasser        & \num{0,0727} \\
  Quecksilber   & \num{0,4650} \\
  Ethanol       & \num{0,0225}
 \end{tabular}
 \caption{Oberfl�chenspannungen einiger Fl�ssigkeiten bei $20\degC$}
\end{table}

\end{document}
