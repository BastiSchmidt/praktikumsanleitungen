\documentclass[platz]{tudphygp}
\usepackage{tudphymd,mhchem}

\versuch{Optische Abbildung}{OA}

\begin{document}
\maketitle

\section*{Aufgabenstellung}

\begin{enumerate}

\item \textbf{Bestimmung der Brennweite $f'_s$ einer d�nnen Sammellinse (S) aus der einfachen Abbildung eines Objektes:}
	\begin{itemize}
	\item F�r \emph{drei} unterschiedliche Schirmeinstellungen soll die Linse je \emph{dreimal} so eingestellt werden, dass das Objekt 
	scharf abgebildet wird. Notieren Sie sich jeweils die Objekt- ($l_1$), die Linsen- ($l_2$) und die Schirmposition ($l_3$).
	\item Bilden Sie den Mittelwert der Linsenposition $l_2$ jeweils f�r die drei unterschiedlichen Schirmpositionen $l_3$.
	\item Aus diesen Daten bestimmen Sie die Gegenstandsweite $a$ und die Bildweite $a'$. Mit Hilfe der allgemeinen 
	Abbildungsgleichung berechnen Sie die Brennweite $f'_s$ der Sammellinse.
	\item Bestimmen Sie den absoluten Fehler $\triangle f'_s$ durch Fehlerfortpflanzung der verwendeten Abbildungsgleichung.
	\end{itemize}
	
\item \textbf{Bestimmung der Brennweite $f'_z$ einer d�nnen Zerstreuungslinse (Z) durch Kombination mit einer Sammellinse 
und dem Hauptebenenabstandes $d \approx 0$ mit Hilfe des \textsc{Bessel}-Verfahrens:}
	\begin{itemize}
	\item Verschieben Sie den Schirm ganz nach hinten. Stecken Sie die Sammel- und Zerstreuungslinse m�glichst dicht in die Halterung. 
	F�r \emph{eine} feste Schirmstellung soll die Linse je \emph{dreimal} so eingestellt werden, dass das Objekt jeweils "`vorne"' 
	und "`hinten"' scharf abgebildet wird. Notieren Sie sich jeweils die Objekt- ($l_1$), die Linsen- ($l_{2,v,h}$) und die 
	Schirmposition ($l_3$).
	\item Bilden Sie den Mittelwert der Linsenpositionen $l_{2,v}$ bzw. $l_{2,h}$ und bestimmen daraus den Abstand beider 
	Linsenpositionen $l$. Berechnen Sie die Brennweite $f'$ des Gesamtsystems.
	\item Bestimmen Sie aus der allgemeinen Abbildungsgleichung die Brennweite $f'_z$ der Zerstreuungslinse.
	\end{itemize}

\item \textbf{Bestimmung der Brennweite $f'$ einer Linsenkombination unter Verwendung des Abbildungsma�stabes (\textsc{Abbe}-Verfahren):}
	\begin{itemize}
	\item Stecken Sie die Sammel- und Zerstreuungslinse so in den Halter, dass der Abstand der Hauptebenen $d = 50~\mathrm{mm}$ betr�gt. 
	Beim Einbau auf die optische Bank soll die Sammellinse Richtung Lampe zeigen. Bestimmen Sie f�r \emph{f�nf} verschiedene 
	Schirmstellungen \emph{jeweils eine} "`vordere"' und "`hintere"' Position des Linsenhalters, wo das Objekt scharf abgebildet wird. 
	W�hlen Sie die Schirmabst�nde so, dass die Bildgr��en sich deutlich unterscheiden. Notieren Sie sich jeweils die Objekt- ($l_1$), 
	die Linsen- ($l_{2,v,h}$) und die Schirmposition ($l_3$), sowie die Gegenstands- und Bildgr��en $y$ und $y'_{v,h}$.
	\item Berechnen Sie f�r die f�nf unterschiedlichen Schirmstellungen die Entfernungen $b_{v,h} = l_{2,v,h} - l_1$ und $b'_{v,h} = l_3 - l_{2v,h}$ 
	zur Ablesemarke des Linsenhalters. Bestimmen Sie aus Objektgr��e $y$ und Bildgr��en $y'_{v,h}$ die Abbildungsma�st�be $\frac{1}{\beta_{v,h}}$ 
	und $\beta_{v,h}$.
	\item Zeichnen Sie in ein Diagramm die Abst�nde $b$ und $b'$ als Funktion der Abbildungsma�st�be $\frac{1}{\beta}$ bzw. $\beta$ ein, 
	d.h. $b(\frac{1}{\beta})$ und $b'(\beta)$. W�hlen Sie den Wertebereich der $\frac{1}{\beta}$- und $\beta$-Achse vom gemessenen 
	Minimalwert von $\frac{1}{\beta}$ bzw. $\beta$. Bestimmen Sie die Abst�nde $h$ und $h'$ von der Ablesemarke zu den Hauptebenen $H$ und $H'$ 
	aus dem Achsenwert f�r $b(1)$ bzw. $b'(1)$. Bestimmen Sie die resultierenden Brennweiten aus $f' = b(0) - h$ bzw. $f' = b'(0) + h'$. 
	Der Mittelwert beider Brennweiten ergibt die gesuchte Brennweite $f'$ der Linsenkombination.
	\item Vergleichen Sie die experimentell bestimmte Brennweite $f'$ mit der theoretisch zu erwartenden Brennweite der Linsenkombination.
	\end{itemize}
	
\item \textbf{\emph{(F�r Physikstudenten)} Bestimmung der sph�rischen Aberration einer extrem dicken Sammellinse als Brennweiten�nderung 
in Abh�ngigkeit von dem Abstand zur optischen Achse mittels \textsc{Bessel}-Verfahren:}
	\begin{itemize}
	\item Ersetzen Sie den Linsenhalter durch die extrem dicke Sammellinse, wobei die planparallele Seite in Richtung Schirm zeigen soll, 
	und realisieren Sie die scharfe Abbildung des Objekts.
	\item Bestimmen Sie mittels des \textsc{Bessel}-Verfahrens die Brennweite $f'_0$ der Sammellinse mit Lochblende, die Randstrahlen 
	ausblendet. Der endliche Abstand der Hauptebenen wird dabei vernachl�ssigt.
	\item Bestimmen Sie mittels des \textsc{Bessel}-Verfahrens die Brennweite $f' = f'(r^2)$ in Abh�ngigkeit vom Radius $r$ der 
	unterschiedlichen Zonenblende. Wiederum wird der endliche Abstand der Hauptebenen vernachl�ssigt.
	\item Tragen Sie alle Brennweiten in ein Diagramm $f' = f'(r^2)$ ein und bestimmen Sie die Ausgleichsgerade.
	\item Warum treten an den R�ndern des Objektes Farbs�ume auf?
	\end{itemize}
	
	
\end{enumerate}
\end{document}