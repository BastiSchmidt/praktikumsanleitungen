\documentclass[platz]{tudphygp}
\usepackage{tudphymd,mhchem,listliketab}

\versuch{Elektrolytischer Trog}{ET}

\begin{document}
\maketitle

\section*{Aufgabenstellung}
\begin{enumerate}
 \item Nehmen Sie den Potentiallinienverlauf folgender Elektrodenanordnung auf:
 \begin{enumerate}
  \item Rohrkondensator
  \item Doppelleitung
  \item Plattenkondensator
 \end{enumerate}
 \item Konstruieren Sie mit Hilfe der Potentiallinien die Verschiebungslinien!
 \item Ermitteln Sie die Kapazit�t von den Elektrodenanordnung a) und b)! Vergleichen Sie diese mit den theoretischen Werten!
\end{enumerate}

\section*{Hinweise}
\begin{description}
 \item[Aufgabe 1:] Ordnen Sie die Elektroden so an, dass die entstehenden Symmetrieachsen Ihre Messungen erleichtern! Bauen Sie 
 eine Br�ckenschaltung auf! Der eine Zweig ist ein Dekadenwiderstand mit 10 festen Teilerverh�ltnissen. Der zweite Zweig 
 ist der elektrolytische Trog mit der Messspitze. Durch Verschiebung der Messspitze wird die Br�cke jeweils abgeglichen. 
 Die Abgleichpositionen sind sofort auf Millimeterpapier einzutragen. Die Messungen werden erleichtert, wenn der ungef�hre Verlauf 
 der �quipotentiallinien vorher �berlegt wird! Nach M�glichkeit alle 9 �quipotentiallinien messen. (Nicht immer sind alle von der 
 Geometrie der Einrichtung her messbar!)\\
 \textbf{Achtung!:} Durch die tonfrequente Signalquelle wird der Einfluss galvanischer Spannungen vermieden. Ein �bergangswiderstand 
 zwischen Elektrode und Elektrolyt kann die Messung aber stark beeinflussen. Sorgen Sie daf�r, dass die Elektroden vor der Messung 
 nicht korrodiert sind.
\item[Aufgabe 2:] Zur Berechnung von $C$ sind auch die nicht gemessenen �quipotentiallinien mitzuz�hlen.
\end{description}

\section*{Ger�te und Hilfsmittel}
\storestyleof{itemize}
\begin{listliketab}
 \begin{tabular}{ll}
  Elektrolyt: & Ammoniumsulfatl�sung (2\%)\\
  Spannungsquelle: & Frequenzgenerator (Frequenz von etwa $\SI{800}{Hz}$ w�hlen!)\\
  Abgleichger�t: & Kopfh�rer\\
  Millimeterpapier A4 &
 \end{tabular}
\end{listliketab}

\textbf{Bitte nach Beendigung des Versuches die Elektroden aus dem Elektrolyten nehmen und unter klarem Wasser absp�len!}

\end{document}
