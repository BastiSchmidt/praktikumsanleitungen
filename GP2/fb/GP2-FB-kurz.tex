\documentclass[platz]{tudphygp}
\usepackage{tudphymd,mhchem}

\versuch{Fraunhoferbeugung}{FB}

\begin{document}
\maketitle

\bigskip

\begin{tabular}{ll}
\textsc{Achtung:} & \emph{Beim Arbeiten mit einem Laser niemals in das Laserlicht hineinschauen!} \\
				  & \emph{Dies gilt ebenso f�r indirektes Laserlicht wie z.B. Reflexion an Metallfl�chen.}\\
\end{tabular}

\bigskip

\section*{Aufgabenstellung}

\begin{enumerate}
 \item Machen Sie sich mit der optischen Bank vertraut! Testen Sie die Justierfreiheitsgrade der 
 Einzelkomponenten und die Einstellm�glichkeiten der Blenden.
 
 \emph{Hinweis:\\
 Die Justierung des Lasers und der Aufweitungsoptik sind keinesfalls zu verstellen. (In der hinteren 
 Brennebene der ersten Linse des Strahlaufweiters ist eine Modenblende zentriert, die Beugungserscheinungen 
 durch Staubpartikel und interne Blenden des Lasers herausfiltert. Eine Justierung dieser Blende ist sehr zeitraubend 
 und aufw�ndig!)}

 \item Stellen Sie die f�r den Versuch notwendigen Komponenten auf der optischen Bank auf um den 
 Beugungsstrahlengang zu realisieren! Stellen Sie dabei sicher, dass die CCD-Zeile in der hinteren Brennebene 
 der Linse positioniert ist.
 \item �berpr�fen Sie anhand einiger Gitter gleichen Typs mit unterschiedlichen Gitterkonstanten die Eigenschaften 
 der Fouriertransformation. Beginnen Sie mit dem Kosinusgitter als Elementarfall.
 
 Was passiert mit dem Beugungsbild beim Verschieben des Beugungsobjekts entlang der optischen Achse (axial) und 
 senkrecht zum Strahlengang (lateral)? 
 
 Wie �ndert sich f�r einen festen Gittertyp das Beugungsbild bei einer Vergr��erung (Verkleinerung) der Gitterkonstante?
 
 Welchen Einfluss hat die Breite des beleuchteten Gitterbereiches auf die Beugungsreflexe? Reduzieren Sie dazu mittels 
 Spaltblende und Irisblende die Gr��e der Beleuchtung.
 \item Bestimmen Sie mithilfe der Fraunhoferbeugung die Gitterkonstanten zweier Gitter unterschiedlicher Art. 
 Vergleichen Sie Ihre Messungen mit den am Lichtmikroskop bestimmten Gitterkonstanten. 
\end{enumerate}

\textbf{Weitere Hinweise:} \\
\begin{itemize}
 \item Die Brennweite der Sammellinse betr�gt $300~\mathrm{mm}$
 \item Die CCD-Zeile umfasst 5304 Pixel mit einer Pixelgr��e von $7~\mathrm{\mu m}$.
\end{itemize}

\end{document}

